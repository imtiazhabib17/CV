%!TEX TS-program = xelatex
%!TEX encoding = UTF-8 Unicode
% Awesome CV LaTeX Template for CV/Resume
%
% This template has been downloaded from:
% https://github.com/posquit0/Awesome-CV
%
% Author:
% Claud D. Park <posquit0.bj@gmail.com>
% http://www.posquit0.com
%
% Template license:
% CC BY-SA 4.0 (https://creativecommons.org/licenses/by-sa/4.0/)
%

%-------------------------------------------------------------------------------
% CONFIGURATIONS
%-------------------------------------------------------------------------------
% A4 paper size by default, use 'letterpaper' for US letter
\documentclass[11pt, a4paper]{awesome-cv}

\usepackage{datetime2}

\usepackage{xcolor} % Include the xcolor package

\usepackage{hyperref}
% Configure page margins with geometry
\geometry{left=1.6cm, top=1.0cm, right=1.6cm, bottom=1.9cm, footskip=.5cm}

% Specify the location of the included fonts
\fontdir[fonts/]

% Color for highlights
% Awesome Colors: awesome-emerald, awesome-skyblue, awesome-red, awesome-pink, awesome-orange
%                 awesome-nephritis, awesome-concrete, awesome-darknight
\colorlet{awesome}{awesome-red}
% Uncomment if you would like to specify your own color
% \definecolor{awesome}{HTML}{CA63A8}

% Colors for text
% Uncomment if you would like to specify your own color
% \definecolor{darktext}{HTML}{414141}
% \definecolor{text}{HTML}{333333}
% \definecolor{graytext}{HTML}{5D5D5D}
% \definecolor{lighttext}{HTML}{999999}

% Set false if you don't want to highlight section with awesome color
\setbool{acvSectionColorHighlight}{true}

% If you would like to change the social information separator from a pipe (|) to something else
\renewcommand{\acvHeaderSocialSep}{\quad\textbar\quad}


%-------------------------------------------------------------------------------
%	PERSONAL INFORMATION
%	Comment any of the lines below if they are not required
%-------------------------------------------------------------------------------
% Available options: circle|rectangle,edge/noedge,left/right
% \photo[rectangle,edge,right]{./examples/profile}
\name{Md. Imtiaz Habib}{}
    %\position{AI{\enskip\cdotp\enskip}NLP{\enskip\cdotp\enskip}Computer Vision {\enskip\cdotp\enskip}Autonomous Swarm{\enskip\cdotp\enskip} Graph Neural Network{\enskip\cdotp\enskip}Algorithm}
%\mobile{(+88) 1609727343}
\email{\textcolor{awesome-skyblue}{imtiazhabib17@gmail.com}}
%\github{imtiazhabib17}
%\linkedin{posquit0}
% \gitlab{gitlab-id}
% \stackoverflow{SO-id}{SO-name}
% \twitter{@twit}
% \skype{skype-id}
% \reddit{reddit-id}
% \medium{madium-id}
\googlescholar{H1vsVtwAAAAJ} {\textcolor{awesome-skyblue}{Google Scholar}}
%% \firstname and \lastname will be used
% \googlescholar{googlescholar-id}{}
% \extrainfo{extra informations}

%\quote{``Be the change that you want to see in the world."}


%-------------------------------------------------------------------------------
\begin{document}
% Print the header with above personal informations
% Give optional argument to change alignment(C: center, L: left, R: right)
\makecvheader[C]

% Print the footer with 3 arguments(<left>, <center>, <right>)
% Leave any of these blank if they are not needed

\makeatletter
\newcommand{\mydate}{\@ifundefined{mydate}{\today}{\mydate}}
\makeatother

\renewcommand{\mydate}{\ifcase\month\or
    January\or
    February\or
    March\or
    April\or
    May\or
    June\or
    July\or
    August\or
    September\or
    October\or
    November\or
    December\fi
    \space\number\year}
    
\makecvfooter
  {\mydate} % This will display the month name and year
  {Md. Imtiaz Habib~~~· Curriculum Vitae}
  {\thepage}





%-------------------------------------------------------------------------------
%	CV/RESUME CONTENT
%	Each section is imported separately, open each file in turn to modify content
%-------------------------------------------------------------------------------
%-------------------------------------------------------------------------------
%	SECTION TITLE
%-------------------------------------------------------------------------------
\cvsection{Research Interests}

\begin{cventries}
  \cventry
    {AI \hspace{0.4em}•\hspace{0.4em}  NLP  \hspace{0.4em}•\hspace{0.4em}  Computer Vision  \hspace{0.4em}•\hspace {0.4em}Graph Neural Network \hspace{0.4em}•\hspace{0.4em} Autonomous Swarm \hspace{0.4em}•\hspace{0.4em}  Algorithm}
    {} % Institution
    {} % Location
    {} % Date(s)\hspace{0.5em}
    {} % Extra braces as defined in the class file
\end{cventries}
\vspace{-\baselineskip} % Add this line to remove the white space







%-------------------------------------------------------------------------------
%	Subsection

%\cvsubsection{\textcolor{awesome-skyblue}{\fontsize{12}{12}\textnormal{\bodyfont\bfseries Interests:}} \fontsize{9}{10}\selectfont  AI • NLP • Computer Vision • Autonomous Swarm • Graph Neural Network • Algorithm} \medskip

%\cvsubsection{\textcolor{awesome-skyblue}{\fontsize{12}{12}\textnormal{\bodyfont\bfseries Skills:}} \fontsize{9}{10}\selectfont  Python • C++ • TensorFlow • Java • C-Sharp • DotNeT • SQL • LaTex} \midskip


%-------------------------------------------------------------------------------
%	SECTION TITLE
%-------------------------------------------------------------------------------
\cvsection{Education}
%------------------------------------
\begin{cventries}

%---------------------------------------------------------
  \cventry
    {B.S. in Computer Science and Engineering} % Degree
    {American International University- Bangladesh} % Institution
    {Dhaka, BD} % Location
    {19 Mar. 2023} % Date(s)
    {  
    \begin{cvitems} % Description(s) of tasks/responsibilities
        \item {Second Major: Software Engineering}
        \item {Thesis: Automated Caption Generator Using NLP}
        \item {Advisor: \textit{Prof. Md. Kishor Morol}}
        \item {CGPA: 3.71/4.00}
    \end{cvitems}
    } 
\end{cventries}

%-------------------------------------------------------------------------------
%	SECTION TITLE
%-------------------------------------------------------------------------------
\cvsection{Work Experience}


%-------------------------------------------------------------------------------
%	CONTENT
%-------------------------------------------------------------------------------
\begin{cventries}

%---------------------------------------------------------
  \cventry
    {Software Engineer Intern} % Job title
    {Mediasoft Data System Limited} % Organization
    {Dhaka, BD} % Location
    {Aug. 2022 - Jan. 2023} % Date(s)
    {
      \begin{cvitems} % Description(s) of tasks/responsibilities
        \item {Built software application, mostly for retail industry.}
        \item {Mainly used C-Sharp, DotNet framework and react for building applications.}
      \end{cvitems}
    }

    \cventry
    {ML Engineer Intern} % Job title
    {Zantrik} % Organization
    {Dhaka, BD} % Location
    {Mar. 2023 - Current} % Date(s)
    {
      \begin{cvitems} % Description(s) of tasks/responsibilities
        \item {Worked on building and maintaing the chatbot called Autobot.}
      \end{cvitems}
    }

%---------------------------------------------------------
\end{cventries}

\cvsection{Research Articles}

\cvsubsection{Thesis}
\vspace{0.5\baselineskip} % Add a smaller vertical space


\begin{mycventry}
    {\textcolor{red}{\fontsize{8.4}{8.4}\textnormal{\hspace{0.7em} \hspace{0.7em}Md. Imtiaz Habib,}} \textcolor{graytext}{\fontsize{8.4}{8.4}\textnormal{Md. Mehedi Hasan Zamil, Syeda Meherunnasha Mim, Talha Zubayer.}}\textcolor{text}{\fontsize{8.4}{8.4}\textbf{Automated Caption Generator Using NLP. }}\textcolor{graytext}{\fontsize{8.4}{8.4}\textnormal{AIUB, 2022. \textit{ Advisor: Prof. Md. Kishor Morol.}}} \href{https://www.researchgate.net/publication/375183559_Automated_Caption_Generator_Using_NLP}{\textcolor{red}{\fontsize{8}{8}\textnormal{   [Paper]}}} \vspace{-\baselineskip}} 
    {}
\end{mycventry}
%-------------------------------------------------------------------------------

\cvsubsection{Research Papers}
\vspace{0.5\baselineskip} % Add a smaller vertical space


\begin{mycventry}
    {\textcolor{graytext}{\fontsize{8,4}{8.4}\textnormal{\hspace{0.7em} \hspace{0.7em}Arafat Islam and }} \textcolor{red}{\fontsize{8.4}{8.4}\textnormal{Md. Imtiaz Habib.}}\textcolor{text}{\fontsize{8.4}{8.4}\textbf{ Antimagic Labeling of Graphs Using Prime Numbers.}}\textcolor{graytext}{\fontsize{8}{8}\textnormal{ Under Review. \textit{ Advisor: Prof. Dr. Md. Manzurul Hasan. }}}\href{https://www.researchgate.net/publication/375238784_ANTIMAGIC_LABELING_OF_GRAPHS_USING_PRIME_NUMBERS?channel=doi&linkId=6544c01cb1398a779d5887c2&showFulltext=true}{\textcolor{red}{\fontsize{8}{8}\textnormal{   [Paper]}}}\vspace{-\baselineskip}}
    {}
\end{mycventry}


\begin{mycventry}
    {\textcolor{red}{\fontsize{8.4}{8.4}\textnormal{\hspace{0.7em} \hspace{0.7em}Md. Imtiaz Habib,}} \textcolor{graytext}{\fontsize{8,4}{8.4}\textnormal{Abdullah Al Maruf and Md. Jobair Ahmed Nabil.}}\textcolor{text}{\fontsize{8.4}{8.4}\textbf{ An Exploration Into Web Session Security- A Systematic Literature Review.}}\textcolor{graytext}{\fontsize{8}{8}\textnormal{ arXiv preprint, 2022. \textit{ Advisor: Prof. Dr MM Mahbubul Syeed. }}}\href{https://arxiv.org/abs/2310.10687}{\textcolor{red}{\fontsize{8}{8}\textnormal{   [Paper]}}}\vspace{-\baselineskip}}
    {}
\end{mycventry}


\begin{cventries} \vspace{-\baselineskip}

\end{cventries}
%-------------------------------------------------------------------------------
%	SECTION TITLE
%-------------------------------------------------------------------------------
\cvsection{Projects}
\vspace{0.5\baselineskip} % Add a smaller vertical space

\begin{mycventryy}
{•{\textcolor{text}{\fontsize{8.5}{8.5} \textbf{\hspace{0.4em} \hspace{0.4em}Fire Detection From Image and Video Using YOLOv5. }}}\textcolor{graytext}{\fontsize{8}{8}\textnormal{ AIUB, 2022. \textit{ Instructor: Prof. Shakhawat Hossain}}}
  \href{https://github.com/aarafat27/Fire-Detection-Using-YOLOv5.git}{\textcolor{red}{\fontsize{8}{8}\textnormal{ [Source Code]}}}\href{https://arxiv.org/abs/2310.06351}{\textcolor{red}{\fontsize{8}{8}\textnormal{  [Paper]}}}}
\end{mycventryy}

\vspace{4pt} % Add some vertical space

\begin{mycventryy}
  {•{\textcolor{text}{\fontsize{8.5}{8.5} \textbf{\hspace{0.4em} \hspace{0.4em}Signature Verification System using VGG19 By Transfer Learning. }}}\textcolor{graytext}{\fontsize{8}{8}\textnormal{ AIUB, 2022. \textit{ Instructor:  Prof. Shakhawat Hossain}}}
  \href{https://github.com/imtiazhabib17/Signature-Verification-Using-VGG19-By-Transfer-Learning}{\textcolor{red}{\fontsize{8}{8}\textnormal{ [Source Code]}}}\href{https://github.com/imtiazhabib17/Signature-Verification-Using-VGG19-By-Transfer-Learning}{\textcolor{red}{\fontsize{8}{8}\textnormal{  [Paper]}}}}
\end{mycventryy}

\vspace{4pt} % Add some vertical space

\begin{mycventryy}
  {•{\textcolor{text}{\fontsize{8.5}{8.5} \textbf{\hspace{0.4em} \hspace{0.4em}Handwritten Digit Recognition.}}}
  \href{https://github.com/imtiazhabib17/Handwritten-Digits-Recognition.git}{\textcolor{red}{\fontsize{8}{8}\textnormal{ [Source Code]}}}}
\end{mycventryy}




\cvsection{Skills}

\begin{cvskills}
    \cvskill
      {Language and Libraries}
      {C++ \hspace{0.3em}•\hspace{0.3em} Java \hspace{0.3em}•\hspace{0.3em} C-Sharp \hspace{0.3em}•\hspace{0.3em}.Net \hspace{0.3em} • \hspace{0.3em}Python\hspace{0.3em}•\hspace{0.3em}Pytorch\hspace{0.3em} • \hspace{0.3em}TensorFlow \hspace{0.3em} • \hspace{0.3em} NumPy \hspace{0.3em}  • \hspace{0.3em} OpenGL\hspace{0.3em} • \hspace{0.3em} MySql\hspace{0.3em} • \hspace{0.3em} LaTeX }

\end{cvskills}
%%-------------------------------------------------------------------------------
%	SECTION TITLE
%-------------------------------------------------------------------------------
\cvsection{Presentation}


%-------------------------------------------------------------------------------
%	CONTENT
%-------------------------------------------------------------------------------
\begin{cventries}

%---------------------------------------------------------
  \cventry
    {\textit{Instructor: Prof. Nafish Sarwar}} % Job title
    {Civil Plan for ABC Housing Limited  \href{https://youtu.be/iwKYPq6e4lU}{\hspace{0.5em} \textcolor{red}{\fontsize{8}{10}\textbf{Video}}}} % Organization
    {Virtual, AIUB} % Location
    {Mar. 2021} % Date(s)
    {
      \begin{cvitems} % Description(s) of tasks/responsibilities
        \item {Gave a presentation on how to make good design during my undergrad as a part of my Computer Aided Design Course.}
      \end{cvitems}
    }

%---------------------------------------------------------
\end{cventries}

\cvsection{Awards}

\begin{cvhonors}
  \cvhonor{Champion}
  {National High School Debate Competition}
  {Dhaka, BD}
  {2015}
\end{cvhonors}

\begin{cvhonors}
  \cvhonor{Finalist}
  {National High School Debate Competition}
  {Dhaka, BD}
  {2014}
\end{cvhonors}
\cvsection{Standardized Tests}

\begin{cvhonors}
  \cvhonor{Overall: 7.5 \hspace{0.7em}}
  {\textnormal {Listening: 8.5 \hspace{0.4em}•\hspace{0.4em}  Reading: 7.5 \hspace{0.4em}•\hspace{0.4em}  Writing: 6.5  \hspace{0.4em}•\hspace {0.4em}Speaking: 7}}
  {\textcolor{text}{23 June, 2023}}
  {IELTS}
\end{cvhonors}
%-------------------------------------------------------------------------------
%	SECTION TITLE
%-------------------------------------------------------------------------------
\cvsection{References}


%-------------------------------------------------------------------------------
%	CONTENT
%-------------------------------------------------------------------------------
\begin{cventries}

%---------------------------------------------------------
  \cventry
    {\textit{Assistant Professor, AIUB }} % Job title
    {\href{https://cs.aiub.edu/profile/kishor}{Md. Kishor Morol} {\hspace{0.5em} \textcolor{greytext}{\fontsize{8}{10}\textnormal{{\textit{Email: kishoremorolsets@iub.edu.bd}}}}}} % Organization
    {Ithaca, NY} % Location
    {} % Date(s)
    {
      \begin{cvitems} % Description(s) of tasks/responsibilities
        \item {Did my undergrad thesis under prof. Kishor. He's currently doing his PhD at Cornell University}
      \end{cvitems}}
%---------------------------------------------------------
\end{cventries}


%-------------------------------------------------------------------------------
\end{document}
